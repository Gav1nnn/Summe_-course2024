\documentclass[a4paper,12pt]{report} % 定义文档类型

% 中文支持及其他常用包
\usepackage[UTF8]{ctex} % 中文支持
\usepackage{geometry} % 页面边距设置
\geometry{left=2.5cm, right=2.5cm, top=2.5cm, bottom=2.5cm}
\usepackage{graphicx} % 插入图片
\usepackage{amsmath} % 数学公式
\usepackage{hyperref} % 超链接
\usepackage{fancyhdr} % 页眉页脚
\usepackage{titlesec} % 控制标题格式
\usepackage{listings} % 插入代码
\usepackage{setspace} % 行距
\usepackage{xcolor} % 颜色
\usepackage{enumitem} % 列表格式设置

% 页眉页脚设置
\pagestyle{fancy}
\fancyhf{}
\renewcommand{\headrulewidth}{1pt} % 页眉下方线条宽度
\renewcommand{\footrulewidth}{1pt} % 页脚下方线条宽度
\fancyhead[L]{课程作业报告} % 在页眉左侧显示文本
\fancyfoot[C]{\thepage} % 在页脚中间显示页码

% 文档开始
\begin{document}

% 目录页
\tableofcontents
\newpage

% 正文部分
\chapter{标题和章节管理}

\section{一级标题示例}
这是一级标题的内容。接下来是一个二级标题。

\subsection{二级标题示例}
这是二级标题的内容。以下展示文本的基本格式化。

\subsubsection{三级标题示例}
这是三级标题的内容。

\paragraph{四级标题示例}
这是四级标题的内容。

\chapter{正文文本格式}

\section{文本样式}
这是一个普通文本。\\
这是 \textbf{加粗} 的文本。\\
这是 \textit{斜体} 的文本。\\
这是 \underline{带下划线} 的文本。\\
这是 \textcolor{blue}{蓝色} 的文本。

\section{列表}

\subsection{无序列表}
\begin{itemize}
    \item 项目1
    \item 项目2
    \item 项目3
\end{itemize}

\subsection{有序列表}
\begin{enumerate}
    \item 第一点
    \item 第二点
    \item 第三点
\end{enumerate}

\subsection{描述性列表}
\begin{description}
    \item[术语A] 这是术语A的定义。
    \item[术语B] 这是术语B的定义。
\end{description}

\chapter{行距和段距控制}

\section{行距控制}
这是默认行距。\\
\onehalfspacing
这是1.5倍行距。\\
\doublespacing
这是双倍行距。\\
\singlespacing
这是恢复为单倍行距。

\section{段距控制}
\setlength{\parskip}{1em} % 设置段间距为1em,这段文字之前和之后会有1em的段间距。

\chapter{其他常用操作}

\section{脚注和颜色}
这是一个带脚注的文本\footnote{这是脚注的内容},
还可以改变文本颜色,比如 \textcolor{red}{红色文本}。

\end{document}